\section{Types}%
\label{sec:types}

TensorFlock defines a small number of primitives: Bool, Int, Double, and Tensor. They are declared (\emph{id} : \emph{type};) and defined (\emph{id} = \emph{literal};) in two lines, as shown in the following examples. 

\subsection{Booleans}
Bools can take on the boolean literals \emph{True} or \emph{False}, or other boolean expressions.
\begin{lstlisting}
p : Bool;
p = True;
\end{lstlisting}

\subsection{Integers}
Our integer type is declared as Int.
\begin{lstlisting}
x : Int;
x = 44;
\end{lstlisting}

\subsection{Doubles}
When defining literals, Doubles must begin with a digit and contain a period. They can also be defined in scientific notation using the letter 'E' or 'e', followed by the exponent to be raised by 10. 
\begin{lstlisting}
d1 : Double;
d1 = 3.14159265359;

d2 : Double;
d2 = 1.234567e-8;

d3 : Double;
d3 = 1.;
\end{lstlisting}

\subsection{Tensors}
Tensors are declared using a mandatory set of angled brackets that specify the shape ($< >$). The indices of a Tensor must be of type Int, whether it is a variable or literal. The contents of a Tensor, however, must be of type Double. \\
Tensor literals are defined using square brackets, which can be nested for dimensions higher than 1. The contents of a Tensor are accessed with square brackets that contain indices for the correct position. 
\begin{lstlisting}
t : T<a, b>;
t = [[0.0, 1.0], [2.0, 3.0], [4.0, 5.0]];

v : Double;
v = t[0, 1];
\end{lstlisting}


